\section{Auswertung}
\subsection{Untersuchung des Linearverstärkers}
In diesem Teil der Auswertung soll das Verhalten der Verstärkung eines invertierenden Linearverstärkers untersucht werden.
Hierzu werden vier Aufbauten mit unterschiedlichen Widerstandskombinationen vermessen.
Der Aufbau erfolgt nach Abbildung \ref{invert} mit den folgenden Widerstandspaaren aus Tabelle \ref{Tab_2}.
Der Verstärkungsfaktor $V_\text{ideal}$, allein aus dem Widerstandsverhältnis für einen idealen Operationsverstärker, wird zum Vergleich der erhaltenen Messdaten nach \eqref{eq:V_linear} berechnet; das negative Vorzeichen ergibt sich aus dem invertierenden Verhalten des Verstärkers.
\begin{table}[]
\centering
\begin{tabular}{c|ccc}
&$R_1\,[\si{\kilo\ohm}]$&$R_N\,[\si{\kilo\ohm}]$&Verstärkungsfaktor $V_\text{ideal}$\\
\hline
Verstärker 1 & 100 & 10  &$-1/10$\\
Verstärker 2 & 100 & 1   &$-1/100$\\
Verstärker 3 & 10  & 0,5 &$-1/20$\\
Verstärker 4 & 10  & 33  &$-3{,}3$
\end{tabular}
\caption{Widerstandsparameter der äußeren Beschaltung $R_1$ und $R_N$ der Grundschaltung des invertierenden Verstärkers mit der daraus berechneten Verstärkung $V_\text{ideal}$.}
\label{Tab_2}
\end{table}
Gemessen wurde die Ausgangsspannung $U_A$ in Abhängigkeit der Frequenz $\nu$.
Ermittelt werden soll die Grenzfrequenz $\nu_\text{Grenz}$, bei der die Verstärkung auf einen Faktor $\frac{1}{\sqrt{2}}$ abgefallen ist. Die Verläufe der Verstärkung für die vier Widerstandspaare sind in den Abbildungen \ref{linear1}, \ref{linear2}, \ref{linear3} und \ref{linear4} zusehen.

Für die Verstärker 1, 2 und 3 ist der Absolutbetrag des Logarithmus des Spannungsverhältnis $\frac{U_A}{U_1}$ verwendet worden, da hier eine Verstärkung $|V'|<1$ vorliegt, sodass der direkte Vergleich mit dem Graphen des Verstärkers 4 nicht ohne Weiteres möglich ist. Des Weiteren ist darüber möglich, die Grenzfrequenz $\nu_\text{Grenz}$ direkt zu berechnen, ohne den Definitionsbereich des Logarithmus zu überschreiten.
Die Annahme, dass die Verstärkung direkt proportional zum Verhältnis der Widerstände der äußeren Beschaltung ist, rührt aus der Betrachtung des Operationsverstärkers als ideales Bauteil, sodass sich die Verstärkung nach \eqref{eq:V_linear} ergibt.\\
Für alle vier Beschaltungen wird eine Eingangsspannung von $U_1=\SI{3}{V}$ verwendet.
Zur Bestimmung der Grenzfrequenz $\nu_\text{Grenz}$ und des Verstärkung-Bandbreite-Produkts $V_0\cdot\nu_\text{Grenz}$ wird zunächst der konstante Verlauf der experimentellen Kennlinie untersucht. Aufgrund des anzunehmenden konstanten Wertes, wird jeweils der Mittelwert der in den Graphen als orange eingezeichneten Messwerte $\bar{V}_0$ gebildet.
Für die Bestimmung der Grenzfrequenz $\nu_\text{Grenz}$ wird eine Ausgleichsrechnung der Form
\begin{equation}
  f(x)=mx^b
\end{equation}
erstellt, um die Steigung des Plots zu bestimmen.
Hierüber lässt sich mittels
\begin{equation}
  \nu_\text{Grenz}=\exp(\ln(V_0/\sqrt{2})-b/m)
\end{equation}
die Grenzfrequenz bestimmen sowie schließlich das Verstärkung-Bandbreite-Produkt $V_0\cdot\nu_\text{Grenz}$, welches im Idealfall konstant sein soll. Die erhaltenen Ausgleichswerte sowie die daraus bestimmten Kenngrößen sind in Tabelle \ref{Werte_fit} zu finden. Aufgrund des Verlaufs des Graphen in Abbildung \ref{linear1}, für den der Verstärkungsteil auch nach Betragsbildung nicht eindeutig zugeordnet werden kann, bleibt dieser als weiterer Vergleichswert fragwürdig.

\begin{table}[H]
  \small
\centering
\begin{tabular}{c|ccccc}
&$m\,[\frac{1}{\si{\ln(\mathrm{Hz)}}}]$ & $b$ & $\ln(\bar{V}_0)$ & $\nu_\text{Grenz}\,[\mathrm{Hz}]$ & GBP$\,[\mathrm{Hz}]\, (V_0\cdot\nu_\text{Grenz})$\\
\hline
Verstärker 1 & $\si{0,776\pm0,008}$ & $-\si{7,97\pm0,10}$ & $\si{2,275}$ & $(5{,}3 \pm 0{,}9) \cdot 10^{4}$ & $(1{,}21 \pm 0{,}21) \cdot 10^{5}$\\
Verstärker 2 & $-\si{0,756\pm0,028}$ & $\si{14,1\pm0,4}$ & $\si{4,423}$ & $(2{,}8 \pm 2{,}3) \cdot 10^{7}$ & $(1{,}2 \pm 1{,}0) \cdot 10^{8}$\\
Verstärker 3 & $-\si{0,248\pm0,009}$ & $\si{4,69\pm0,07}$ & $\si{2,989}$ & $(8{,}0 \pm 6{,}0) \cdot 10^{6}$ & $(2{,}5 \pm 1{,}6) \cdot 10^{7}$\\
Verstärker 4 & $-\si{0,815\pm0,010}$ & $\si{8,86\pm0,11}$ & $\si{0,143}$ & $(8{,}8 \pm 1{,}9) \cdot 10^{5}$ & $(1{,}25 \pm 0{,}27) \cdot 10^{5}$\\
\end{tabular}
\caption{Ergebnisse der Ausgleichsrechnung und der mittels dieser berechneten Werte der Grenzfrequenz $\nu_\text{Grenz}$ und des Verstärkung-Bandbreite-Produkts GBP.}
\label{Werte_fit}
\end{table}

\begin{figure}[h]
  \centering
  \includegraphics[width=0.60\textwidth]{Linearverstaerker_Anneke/A1.pdf}
  \caption{Graphischer Verlauf des Verstärkungsfaktors $V'$ in Abhängigkeit der Frequenz $\nu$ für den ersten beschalteten Linearverstärker. Das Widerstandsverhältnis beträgt $\frac{R_N}{R_1}=\frac{\SI{10}{k\ohm}}{\SI{100}{k\ohm}}$. Die als konstant anzunehmenden Messwerte sind orangefarben markiert worden; die des Verstärkungsverlaufs grün. Für die Auswertung nicht verwendeten Messwerte sind schwarz gezeichnet. Die in blau gezeichnete lineare Ausgleichsrechnung soll die Verstärkung annähern.}
  \label{linear1}
\end{figure}
\begin{figure}[h]
  \centering
  \includegraphics[width=0.60\textwidth]{Linearverstaerker_Anneke/A2.pdf}
  \caption{Graphischer Verlauf des Verstärkungsfaktors $V'$ in Abhängigkeit der Frequenz $\nu$ für den zweiten beschalteten Linearverstärker. Das Widerstandsverhältnis beträgt $\frac{R_N}{R_1}=\frac{\SI{1}{k\ohm}}{\SI{100}{k\ohm}}$. Die als konstant anzunehmenden Messwerte sind orangefarben markiert worden; die des Verstärkungsverlaufs grün. Für die Auswertung nicht verwendeten Messwerte sind schwarz gezeichnet.  Die in blau gezeichnete lineare Ausgleichsrechnung soll die Verstärkung annähern.}
  \label{linear2}
\end{figure}
\begin{figure}[h]
  \centering
  \includegraphics[width=0.60\textwidth]{Linearverstaerker_Anneke/A3.pdf}
  \caption{Graphischer Verlauf des Verstärkungsfaktors $V'$ in Abhängigkeit der Frequenz $\nu$ für den dritten beschalteten Linearverstärker. Das Widerstandsverhältnis beträgt $\frac{R_N}{R_1}=\frac{\SI{0,5}{k\ohm}}{\SI{10}{k\ohm}}$. Die als konstant anzunehmenden Messwerte sind orangefarben markiert worden; die des Verstärkungsverlaufs grün. Für die Auswertung nicht verwendeten Messwerte sind schwarz gezeichnet.  Die in blau gezeichnete lineare Ausgleichsrechnung soll die Verstärkung annähern.}
  \label{linear3}
\end{figure}
\begin{figure}[h]
  \centering
  \includegraphics[width=0.60\textwidth]{Linearverstaerker_Anneke/A4.pdf}
  \caption{Graphischer Verlauf des Verstärkungsfaktors $V'$ in Abhängigkeit der Frequenz $\nu$ für den vierten beschalteten Linearverstärker. Das Widerstandsverhältnis beträgt $\frac{R_N}{R_1}=\frac{\SI{33}{k\ohm}}{\SI{10}{k\ohm}}$. Die als konstant anzunehmenden Messwerte sind orangefarben markiert worden; die des Verstärkungsverlaufs grün.  Die in blau gezeichnete lineare Ausgleichsrechnung soll die Verstärkung annähern.}
  \label{linear4}
\end{figure}
\clearpage
Für den nächsten Schritt wird die gemessene Phase zwischen der Eingangs- und der Ausgangsspannung $\varphi$ gegen die eingestellte Frequenz $\nu$ aufgetragen. Hierdurch soll die Phasenverschiebung zwischen den beiden Signalen bei sich ändernder Frequenz untersucht werden. Zu sehen ist dieser Verlauf in Abbildung \ref{phase}.
Erkennbar ist, dass die beiden Signale bei niedrigen Frequenzen einer Phasenverschiebung um $\SI{180}{^\circ}$ unterliegen; mit steigender Frequenz wird diese abgebaut. Dieses Abbauen zeigt, dass der Verstärker ab einer gewissen kritischen Frequenz nicht mehr linear verstärkt.
Durch Augenmaß kann hier eine kritische Frequenz von $\SI{10}{kHz}$ bestimmt werden.
\begin{figure}[h]
  \centering
  \includegraphics[width=0.65\textwidth]{Linearverstaerker_Anneke/phase.pdf}
  \caption{Phasenverschiebung zwischen der Ausgangs- und der Eingangsspannung $\varphi$ in Abhängigkeit der eingestellten Frequenz $\nu$. Es zeigt sich eine anfängliche Verschiebung um $\SI{180}{^\circ}$, die nach etwa $\SI{10}{kHz}$ abnimmt. Diese Grenze ist durch eine orangefarbene Linie gekennzeichnet.}
  \label{phase}
\end{figure}

\subsection{Untersuchung des Umkehr-Integrators und Differentiators}
Es wird untersucht, in welchem Bereich die theoretischen Zusammenhänge zwischen Ausgangsspannung und Frequenz erfüllt sind. Hierzu wird erneut die Ausgangsspannung für beide Aufbauten doppeltlogarithmisch gegen die Frequenz aufgetragen und die Steigung der auftretenden Gerade durch eine Ausgleichsrechnung ermittelt. Die Messwerte sowie die Ausgleichskurven sind in den Abbildungen \ref{integrator} und \ref{differentiator} dargestellt. Für beide Schaltungen wurde ein Kondensator mit einer Kapazität von $F=0{,}015\,\si{\micro\farad}$ und ein Widerstand mit $R=10\,\si{\kilo\ohm}$ genutzt. Für die Regressionsfunktion der Form:
\begin{equation}
\log(U_A)=m\cdot\log(\nu)+b\,,
\end{equation}
ergeben sich die Parameter:
\begin{align*}
\text{Umkehr-Differentiator:}\\
m&=\SI{-0,81\pm0,02}{\frac{mV}{Hz}}\\
b&=\SI{5,18\pm0,06}{mV}\\
\text{Umkehr-Integrator:}\\
m&=\SI{1,01\pm0,03}{\frac{mV}{Hz}}\\
b&=\SI{-0,6\pm0,2}{mV}\,.\\
\end{align*}

\begin{figure}[h]
  \centering
  \includegraphics[width=0.65\textwidth]{bilder/diffe.pdf}
  \caption{Die aufgenommenen Messwerte der Ausgangsspannung des Umkehr-Differentiators gegen die Frequenz aufgetragen. Es zeigt sich ein annähernd proportionaler Zusammenhang, der durch eine lineare Ausgleichsrechnung extrapoliert wird.}
\end{figure}

Für den Umkehr-Differentiator ist festzustellen, dass der erwartete Zusammenhang zwischen Ausgangsspannung und Frequenz nur zwischen ca. $100\,\si{\Hz}\.-\.800\,\si{\Hz}$ vorliegt. Beim Umkehr-Integrator hingegen lässt sich kein Bereich finden, in welchem der theoretische Verlauf nicht erfüllt wird. Die integrierenden und differentierenden Eigenschaften der Schaltungen sind in den folgenden Oszilloskopbildern \ref{fig:int} und \ref{fig:diff} gut zu erkennen, hierbei ist das Eingangssignal stets gelb und das Signal der Schaltung grün. Ein Thermodruck für eine Dreiecksspannung als Eingangsspannung für den Differentiator ist zwar aufgenommen worden, ist aber bei der Auswertung des Bildmaterials nicht aufgefunden worden.
\clearpage
Es kann vermutet werden, dass das Oszilloskop diese Aufnahme nicht abgespeichert hat. Die Ergebnisse des Differentiators sehen hingegen der des Integrators nicht erwartungsgemäß aus. Es lässt sich vermuten, dass die Schaltung falsch aufgebaut worden  oder es hier zu schaltungsinternen Fehlern aufgrund kleiner Schwankungen zum Beispiel der Versorgungsspannung gekommen ist.
\begin{figure}[h]
  \centering
  \includegraphics[width=0.65\textwidth]{bilder/integrator.pdf}
  \caption{Die aufgenommenen Messwerte der Ausgangsspannung des Umkehr-Integrators gegen die Frequenz aufgetragen. Es zeigt sich ein annähernd proportionaler Zusammenhang, der durch eine lineare Ausgleichsrechnung extrapoliert wird.}
\end{figure}
\clearpage
\begin{figure}[!ht]
   \centering
   \subfloat[test][Integrierte Sinusspannung.]{\includegraphics[width=.42\textwidth]{bilder/integrator1.png}}\quad
   \subfloat[][Integrierte Dreiecksspannung.]{\includegraphics[width=.42\textwidth]{bilder/integrator2.png}}\\
   \subfloat[][Integrierte Rechteckspannung.]{\includegraphics[width=.42\textwidth]{bilder/integrator3.png}}\quad
   \caption{Graphische Aufnahme der Oszilloskopdaten für die Ausgangsspannung in Abhängigkeit unterschiedlicher Eingangssignale für einen Umkehr-Integrator. Die Eingangsspannung ist gelb, die Ausgangsspannung grün dargestellt.}
   \label{fig:int}
\end{figure}
\begin{figure}[!ht]
   \centering
   \subfloat[test][Differenzierte Rechteckspannung.]{\includegraphics[width=.42\textwidth]{bilder/diff1.png}}\quad
   \subfloat[][Differenzierte Sinusspannung.]{\includegraphics[width=.42\textwidth]{bilder/diff2.png}}\\
   \caption{Graphische Aufnahme der Oszilloskopdaten für die Ausgangsspannung in Abhängigkeit unterschiedlicher Eingangssignale für einen Umkehr-Differentiator. Die Eingangsspannung ist gelb, die Ausgangsspannung grün dargestellt. Es zeigen  sich nicht die erwarteten Spannungsverläufe der Ausgangsspannung.}
   \label{fig:diff}
\end{figure}
\clearpage

\subsection{Untersuchung des Schmitt-Triggers}
Die technischen Daten des Aufbaus des Schmitt-Triggers nach Abbildung \ref{schmitti} sind:
\begin{table}[H]
\centering
\begin{tabular}{ccc}
$U_{B,+}=12{,}75 \,\si{\V}$ \\
$U_{B,-}=-13{,}8\,\si{V}$ \\
$U_1=8{,}3\,\si{V}$ \\
$R_1=10\,\si{\kilo\ohm}$ \\
$R_p=33\,\si{\kilo\ohm}$\,. \\
\end{tabular}
\end{table}
Mit Formel \eqref{eq:schmitti} ergibt sich hieraus ein theoretischer Umschlagpunkt von
\begin{equation}
U_{+,\text{theo}}=3{,}86\,\si{\V}\,.
\end{equation}
Der gemessene Schwellwert liegt bei
\begin{equation}
U_{+}=4{,}15\,\si{\V}\,.
\end{equation}
Es liegt somit eine Abweichung von ca. $7\%$ vor.
\clearpage
