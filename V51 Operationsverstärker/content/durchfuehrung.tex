\section{Durchführung}
Die in Kapitel \ref{theorie} genannten Schaltungen werden nach ihren Schaltplänen aufgebaut und folgende Experimente durchgeführt:
\begin{itemize}
\item Bei der Schaltung des invertierenden Verstärkers wird eine Sinusspannung am Eingang angelegt und mit der Ausgangsspannung auf ein Oszilloskop gegeben. Die Verstärkung wird nun auf ihre Frequenzabhängigkeit und Phasenverschiebung hin untersucht. Dies wird für vier Widerstandskombinationen wiederholt.
\item Bei dem Integrator und dem Differentiator wird mithilfe eines Signalgenerators die Art der Eingangsspannung zwischen Sinus-, Dreiecks- und Rechteckssspannung variiert. Die Funktionsweise der Schaltung sowie die Reaktion auf die verschiedenen Formen der Eigangsspannung wird mithilfe des Oszilloskops untersucht. Weiterhin soll ebenso die Frequenzabhängigkeit der Ausgangsspannung untersucht werden.
\item Für eine Schmitt-Trigger-Schaltung soll der Schwellwert der Kippschaltung bestimmt werden. Die Untersuchung erfolgt mit der auf das Oszilloskop gegebenen Ausgangsspannung und einer langsam erhöhten Eingangsspannung.
\end{itemize}
