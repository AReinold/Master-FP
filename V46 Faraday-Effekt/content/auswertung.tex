\section{Auswertung}
\subsection{Bestimmung der maximalen Kraftflussdichte des verwendeten Aufbaus mit Magnetfeld}
Im Folgenden werden die Daten ausgewertet, die aufgenommen wurden, um das maximale Magnetfeld innerhalb des Aufbaus zu bestimmen. Wie bereits in Kapitel \ref{durchfuehrung} beschrieben, ist eine Messreihe mit einer Hall-Sonde durchgeführt worden; die erhaltenen Messwerte sind in Tabelle \ref{tab:messwerte_hall} zu sehen.
Graphisch aufgetragen sind ebendiese in Abbildung \ref{hall}. Zur Bestimmung des Maximums ist eine Ausgleichsrechnung der Form
\begin{equation}
B(z)=az^4+bz^3+cz^2+dz+e
\label{eq:regression}
\end{equation}
für die Messwerte $B(z)>100$ durchgeführt worden. Die Form dieser Funktion ist aufgrund der Symmetrie der Messwerte und der Ähnlichkeit zu quadratischen Funktionen gewählt worden. Diese Ausgleichsrechnung dient als Vergleich zu dem Wert für die maximale Magnetfeldstärke, der ebenso leicht per Augenmaß bestimmt werden kann.
Als Regressionsparameter ergeben sich
\begin{align*}
  a&=-\SI{0,0076 \pm 0,0007}{mTmm^{-4}}\\
  b&=\SI{0,39 \pm 0,06}{mTmm^{-3}}\\
  c&=-\SI{7,6 \pm 0,7}{mTmm^{-2}}\\
  d&=\SI{6,6 \pm 6,0}{mTmm^{-1}}\\
  e&=-\SI{215 \pm 2}{mT}.
\end{align*}
Über die Ableitung der Regressionsfunktion \eqref{eq:regression}
\begin{equation}
\frac{\text{d}}{\text{dz}}B(z)=4a^3+3b^2+2c+d \stackrel{!}{=} 0
\end{equation}
und dem Umstellen nach $z$ folgt ein Maximalwert von
\begin{equation}
 z_\text{max}=\SI{128,24}{mm}.
\end{equation}
Hierüber folgt durch Einsetzen in Formel \eqref{eq:regression}
\begin{equation}
B_\text{max}=\SI{414,5}{mT}.
\end{equation}
\begin{table}[h]
  \footnotesize
  \centering
  \caption{Mit einer Hall-Sonde gemessene Magnetfeldstärke im Bereich des Luftspalts zur Bestimmung des Maximalwerts.}
  \label{tab:messwerte_hall}
  \begin{tabular}{S[table-format=1.1] S S}
    {$z$ / mm} & {$B$ / mT}\\
    \midrule
    100 & 0 \\
    105 & 3 \\
    110 & 14\\
    111 & 20 \\
    112 & 27 \\
    113 & 35 \\
    114 & 48 \\
    115 & 68 \\
    116 & 95 \\
    117 & 135 \\
    118 & 183 \\
    119 & 241 \\
    120 & 304 \\
    121 & 341 \\
    122 & 368 \\
    123 & 382 \\
    124 & 393 \\
    125 & 402 \\
    126 & 407 \\
    127 & 409 \\
    128 & 411 \\
    129 & 410 \\
    130 & 408 \\
    131 & 408 \\
    132 & 404 \\
    133 & 399 \\
    134 & 390 \\
    135 & 378 \\
    136 & 364 \\
    137 & 335 \\
    138 & 308 \\
    139 & 263 \\
    140 & 207 \\
    141 & 163 \\
    142 & 106 \\
    143 & 77 \\
    144 & 54 \\
    145 & 39 \\
    146 & 30 \\
    147 & 21 \\
    148 & 16 \\
    149 & 12 \\
    150 & 8 \\
    151 & 6 \\
    152 & 5 \\
    153 & 3 \\
    154 & 2 \\
    155 & 2 \\
    160 & 0 \\
  \end{tabular}
\end{table}
\clearpage
\begin{figure}[H]
  \centering
  \includegraphics[width=0.8\textwidth]{plots/hall.pdf}
  \caption{Die magnetische Kraftflussdichte $B$ aufgetragen gegen die Längenkoordinate $z$ innerhalb der Magnetspulen. Zur Bestimmung des Maximalwertes von $B$ ist eine Ausgleichsrechnung der Form $B(z)=az^4+bz^3+cz^2+dz+e$ eingezeichnet.}
  \label{hall}
\end{figure}

\subsection{Bestimmung der effektiven Masse}
Für die Faraday-Rotation wird Licht im Wellenlängenbereich von $\SI{1,06}{µm}$ und $\SI{2,65}{µm}$ verwendet.
Die durch Filtern erzeugten neun Wellenlängen sind in den Tabellen \ref{tab:probe1}, \ref{tab:probe2} und \ref{tab:probe3} zu sehen.
Es folgt die Auswertung der drei Proben mit den Parametern
\begin{align*}
d_\text{hochrein}&=\SI{5,11}{mm}\\
d_{N=1,2\cdot10^{18}}&=\SI{1,36}{mm}\\
d_{N=2,58\cdot10^{18}}&=\SI{1,296}{mm}.
\end{align*}
Hierbei ist die erste benannte Probe $d_\text{hochrein}$ die hochreine und die beiden letzten die dotierten $\ce{GaAs}$-Proben mit der Dotierkonzentration $N$.\\
Es werden die Differenzen der beiden abgelesenen Winkel zu
\begin{equation}
  \theta=\frac{1}{2}(\theta_1-\theta_2)
\end{equation}
bestimmt. Diese Winkelgrößen und die daraus abgeleitete Größe $\theta/d$ sind in den Tabellen \ref{tab:probe1}, \ref{tab:probe2} und \ref{tab:probe3} eingetragen.
In der Abbildung \ref{GaAs1} wird $\theta/d$ gegen $\lambda^2$ aufgetragen.\\
Darüber hinaus wird die Differenz der normierten Rotationswinkel $\theta/d$ der beiden dotierten $\ce{GaAs}$-Proben mit den Daten der hochreinen Probe in Abbildung \ref{GaAs2} gebildet und dargestellt.
Für das Ermitteln der effektiven Massen werden zwei lineare Ausgleichsrechnungen der Form
\begin{equation}
  f(\lambda^2)=\theta/d(\lambda^2)=a\lambda^2
\end{equation}
erstellt, aus denen sich die folgenden Parameter ergeben:
\begin{align*}
&\text{hochreine Probe - n-dotiert}\; N=\SI{1.2e18}{\centi\meter^{-3}}:\\
&a_1=\SI{10,49 \pm 4,17e12}{\frac{\mathrm{rad}}{\meter^{-3}}}\\
&\text{hochreine Probe - n-dotiert}\; N=\SI{2.8e18}{\centi\meter^{-3}}:\\
&a_2=\SI{17,33\pm6,43e12}{\frac{\mathrm{rad}}{\meter^{-3}}}\\
\end{align*}
Für diese Berechnung sind die negativen Differenzen der normierten Rotationswinkel außer Acht gelassen worden. Dies rührt aus physikalischen Überlegungen, da diese Werte nicht sinnvoll wären.
Für das Bestimmen der effektiven Massen wird Gleichung \ref{eq:theta2} nach der gesuchten Größe umgestellt, sodass sich
\begin{equation}
m^{*} = \sqrt{\frac{e_0^3}{8\pii^2 \varepsilon_0 c^3} \frac{N B_\text{max}}{n} \frac{1}{a}}
\end{equation}
als Zusammenhang ergibt. Werden nun die Regressionsparameter $a_1$ und $a_2$ aus dem vorherigen Schritt benutzt, so ergeben sich
\begin{align*}
  m_1^{*} &= \SI{0,061 \pm 0,012}{m_\text{e}}\\
  \shortintertext{für die $\ce{GaAs}$-Probe mit $N=\SI{1.2e18}{\centi\meter^{-3}}$ und}
  m_2^{*} &= \SI{0,072 \pm 0,013}{m_\text{e}}
\end{align*}
für die $\ce{GaAs}$-Probe mit $N=\SI{2.8e18}{\centi\meter^{-3}}$, in Einheiten der Ruhemasse $m_\text{e}$ der Elektronen.
Der benötigte Brechungsindex für Galliumarsenid von $n=3,3543$ ist der Literatur entnommen worden \cite{Brechungsindex}.
Es liegen Abweichungen zu dem bestimmten Theoriewert der effektiven Masse $m^{*}=0,067m_\text{e}$ \cite{effmasse} von jeweils $\SI{9,0}{\%}$ und $\SI{7,5}{\%}$ vor.
\begin{table}[h]
  \centering
  \caption{Messwerte und die daraus abgeleiteten Größen $\theta$ und $\theta/d$ der Messung zur n-dotierten $\ce{GaAs}$-Probe mit $N=\SI{1.2e18}{\centi\meter^{-3}}$.}
  \label{tab:probe1}
  \begin{tabular}{S[table-format=1.1] S S S S}
    {$\lambda$ / µm} & {$\theta_1$ / °} & {$\theta_2$ / °} & {$\theta$ / rad} & {$\frac{\theta}{d}$ / rad $m^{-1}$}\\
    \midrule
    1,06 &  168,05 &  158,00 & 0.0877 & 64,49\\
    1,29 &  210,92 &  196,00 & 0.1302 & 95,74\\
    1,45 &  205,92 &  200,75 & 0.0451 & 33,16\\
    1,72 &  161,20 &  158,00 & 0.0279 & 20,52\\
    1,96 &  165,20 &  146,50 & 0.1632 & 119,97\\
    2,156 & 166,20 &  161,00 & 0.0454 & 33,35\\
    2,34 & 200,40 &  187,00 & 0.1169 & 85,96\\
    2,51 & 255,00 &  250,75 & 0.0371 & 27,30\\
    2,65 & 299,00 &  294,00 & 0.0436 & 32,06\\
  \end{tabular}
\end{table}

\begin{table}[h]
  \centering
  \caption{Messwerte und die daraus abgeleiteten Größen $\theta$ und $\theta/d$ der Messung zur n-dotierten $\ce{GaAs}$-Probe mit $N=\SI{2,8e18}{\centi\meter^{-3}}$.}
  \label{tab:probe2}
  \begin{tabular}{S[table-format=1.1] S S S S}
    {$\lambda$ / µm} & {$\theta_1$ / °} & {$\theta_2$ / °} & {$\theta$ / rad} & {$\frac{\theta}{d}$ / rad $m^{-1}$}\\
    \midrule
    1,06 &  279,48 &  275,58 & 0.0340 & 26,24\\
    1,29 &  234,25 &  207,08 & 0.2371 & 174,34\\
    1,45 &  256,50 &  245,15 & 0.0990 & 72,80\\
    1,72 &  296,62 &  270,36 & 0.2292 & 168,50\\
    1,96 &  280,33 &  269,33 & 0.0960 & 70,55\\
    2,156 & 275,35 &  267,03 & 0.0726 & 53,38\\
    2,34 & 307,82 &  290,43 & 0.1518 & 111,58\\
    2,51 & 292,66 &  262,00 & 0.2676 & 196,73\\
    2,65 & 295,05 &  282,05 & 0.1134 & 83,38\\
  \end{tabular}
\end{table}

\begin{table}[h]
  \centering
  \caption{Messwerte und die daraus abgeleiteten Größen $\theta$ und $\theta/d$ der Messung zur hochreinen Probe.}
  \label{tab:probe3}
  \begin{tabular}{S[table-format=1.1] S S S S}
    {$\lambda$ / µm} & {$\theta_1$ / °} & {$\theta_2$ / °} & {$\theta$ / rad} & {$\frac{\theta}{d}$ / rad $m^{-1}$}\\
    \midrule
    1,06 &  243,00 &  226,78 & 0.1415 & 27,69\\
    1,29 &  259,82 &  240,07 & 0.1724 & 33,73\\
    1,45 &  241,25 &  221,62 & 0.1713 & 33,53\\
    1,72 &  175,25 &  166,02 & 0.0805 & 15,76\\
    1,96 &  262,12 &  181,33 & 0.705 & 137,97\\
    2,156 & 178,28 &  171,23 & 0.0615 & 12,04\\
    2,34 & 184,42 &  178,50 & 0.0517 & 10,11\\
    2,51 & 188,42 &  168,33 & 0.1753 & 34,31\\
    2,65 & 203,65 &  156,15 & 0.4145 & 81,12\\
  \end{tabular}
\end{table}
\clearpage
\begin{figure}[H]
  \centering
  \includegraphics[width=0.8\textwidth]{plots/winkel.pdf}
  \caption{Die normierte Faraday-Rotation $\theta/d$ aufgetragen gegen das Quadrat der Wellenlänge $\lambda$ für die n-dotierten und die hochreine $\ce{GaAs}$-Probe.}
  \label{GaAs1}
\end{figure}

\begin{figure}[H]
  \centering
  \includegraphics[width=0.8\textwidth]{plots/linear.pdf}
  \caption{Die Differenz der normierten Faraday-Rotationen der dotierten $\ce{GaAs}$-Proben mit denen der hochreinen $\ce{GaAs}$-Probe aufgetragen gegen $\lambda^2$.}
  \label{GaAs2}
\end{figure}
\clearpage
