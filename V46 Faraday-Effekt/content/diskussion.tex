\section{Diskussion}
Für den ersten Teil der Messung, nämlich der Bestimmung des Maximalwerts des Magnetfeldes, kann festgestellt werden, dass die gewählte Ausgleichsfunktion mit einer vierten Potenz eine sehr gute Näherung liefert. Es ist zu bemerken, dass der Maximalwert genau in der Mitte des Messbereichs liegt und somit ebenso der Mitte des Luftspalts zugeordnet werden kann.
Da die Proben während der Messung recht genau an dieser Mitte angebracht gewesen sind, lässt sich dieser Wert als sehr wahrscheinlich annehmen.\\
Während der Durchführung des zweiten Messabschnitts ist bereits während der Messung aufgefallen, dass das Ablesen am Goniometer nur bedingt genau sein kann. Da das Einstellen des Minimums am Oszilloskop per Augenmaß geschehen ist, hatte dies direkte Auswirkungen auf das Ablesen an dem Winkelmesser. Auch, wenn bei der Durchführung darauf geachtet worden ist, dass die Messung für die beiden Polungen des Magnetfeldes nicht weit voneinander abweichen, ist hier durch die Differenz ein erheblicher Fehler aufgetreten. Somit ergaben sich negative Werte für die Winkeldifferenzen, die für die spätere Auswertung ignoriert worden sind. Da insgesamt pro Probe nur für neun Filter gemessen worden ist, hat dies Auswirkungen, da die Hälfte der Messwerte nicht brauchbar weg gelassen worden sind. Weitere Fehlerquellen können nur vermutet, nicht aber belegt werden, da der Fehler, der durch das Ablesen entstanden ist, zu dominierend ist.
