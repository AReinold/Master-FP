\section{Durchführung}\label{durchfuehrung}
\subsection{Versuchsaufbau}
Der genutzte Versuchsaufbau ist in Abb.\ref{Aufbau} dargestellt. Die Lichtquelle des Aufbaus ist eine Halogen-Lampe, welche ein Spektrum hat, das größtenteils im Infrarotbereich liegt. Das emittierte Licht wird durch eine Linse gebündelt und durch einen Lichtzerhacker in Pulse eingeteilt.
Die Linearpolarisierung des Lichts erfolgt durch ein Glan-Thompson-Prisma, dessen Winkel zum Strahl durch ein Goniometer variiert werden kann. Die Photonen treffen danach auf die scheibenförmige Probe, welche in einem Elektromagneten mit Feldrichtung parallel zur Photonenrichtung platziert ist. Hinter der Probe befindet sich eine Halterung für austauschbare Interferenzfilter. Zur Untersuchung der Rotation der Polarisationsebene wird die Strahlung mit einem zweiten Glan-Thompson-Prisma in zwei senkrecht zueinander polarisierte Teile  zerlegt und die Teilstrahlen werden erneut durch Linsen gebündelt und die Lichtintensität wird mittels Photowiderständen gemessen. Die hohe Rauschspannung, welche an den Photowiderständen auftritt, wird durch die Wechsellichtmethode, welche durch den Lichtzerhacker und einen auf die Frequenz des Zerhackers eingestellten Selektivverstärker gegeben ist, verhindert. Die am Photowiderstand abfallende Wechselspannung wird an Kondensatoren ausgegekoppelt, wobei die Zeitkonstante von einem der Kondensatoren variabel ist. Die Signale beider Photowiderstände werden auf einen Differenzverstärker gegeben und von dort in den Selektivverstärker. Das Signal des Selektivverstärkers wird an einem Oszilloskop angezeigt.
\begin{figure}[H]
  \centering
  \includegraphics[width=1\textwidth]{bilder/aufbau.png}
  \caption{Schematische Darstellung des genutzten Versuchsaufbaus \cite{anleitung}.}
  \label{Aufbau}
\end{figure}
\subsection{Versuchsdurchführung}
\subsubsection{Justierung}
Zu Beginn des Versuchs muss der Aufbau justiert werden. Hierzu werden Probe, Interferenzfilter und die Abdeckungen der Photowiderstandgehäuse entfernt. Der Strahlengang des Lichts wird überprüft und es wird getestet, ob die Veränderungen am Goniometer die erwarteten Intensitätschwankungen an den Teilstrahlen verursachen.\\
Anschließend kann der Lichtzerhacker eingeschaltet und auf eine Frequenz von $\SI{450}{\Hz}$ gestellt werden. Die Mittenfrequenz des Selektivverstärkers wird auf auf den Zerhacker angepasst, indem das Signal eines Photowiderstands am Differenzverstärker gegen ein "Ground"-Signal geschaltet und das resultierende Signal in den Selektivverstärker gegeben wird. Die Frequenz des Selektivverstärkers wird solange variiert bis ein maximales Signal erhalten wird.
\subsubsection{Messung}
Zur Bestimmung des Drehwinkels wird eine Probe und ein Interferenzfilter in den Aufbau eingesetzt und es wird der Elektromagnet auf die maximale Leistung eingestellt. Mittels des Goniometers wird die Intensität der Teilstrahlen so eingestellt, dass das Signal am Oszilloskop sein Minimum erreicht. Ist dies erreicht so wird der am Goniometer eingestellte Winkel aufgezeichnet und das B-Feld umgepolt. Nun werden erneut die beiden Teilstrahlen abgeglichen und der eingestellte Winkel aufgezeichnet. Die Messung wird für drei Proben und jeweils neun Interferenzfilter durchgeführt. Abschließen wird noch die Stärke des B-Feldes mit einer Hall-Sonde bestimmt.
