\section{Theorie}
\subsection{Von Bändern und Massen}%<-jk......darf ich nicht vergessen zu  ändern
Die physikalische Beschreibung von ELektronen in einem Kristall lässt sich am besten durch die Betrachtung der unteren Bandkante des Leitungsbandes annähern. Es lässt sich dann die Elektronenenergie $\epsilon(\vec{k})$, wobei $\vec{k}$ der Wellenzahlvektor ist, in einer Taylorreihe zu:
\begin{equation}
\epsilon(\vec{k})=\epsilon\left(0\right)+\frac{1}{2}\sum_{i=1}^3\left(\frac{\partial\epsilon^2}{\partial k_i^2}\right)_{k=0}k_i^2+...\,,
\end{equation}
entwickeln.
Vergleicht man dies mit einem harmonischen Oszillator mit
\begin{equation}
  \epsilon=\frac{\hbar k^2}{2m}\,,
\end{equation}
so stellt man fest, dass die Größe:
\begin{equation}
m_i^*:=\frac{\hbar^2}{\left(\frac{\partial\epsilon^2}{\partial k_i^2}\right)_{k=0}}\,,
\end{equation}
die Dimension einer Masse hat. Sie wird auch als effektive Masse des Kristallelektrons bezeichnet.
Für hinreichend hohe Symmentrien des Kristalls sind die einzelnen $m_i^*$ alle gleich groß und das Elektron lässt sich wie ein freies Teilchen mit Masse $m_i^*$ behandeln.
\subsection{Zirkulare Doppelbrechung}
Optische Doppelbrechung bezeichnet die Rotation der Polarisationsebene von lineapolarisiertem Licht beim durchqueren eines Mediums. Physikalisch lässt sich dieses Phänomen nachvollziehen, indem man linearpolarisiertes Licht in zwei entgegengesetzt zirkularpolarisierte Komponenten zerlegt
\begin{equation}
E(z)=\frac{1}{2}(E_\text{L}(z)+E_\text{R}(z))\,.
\end{equation}
In einem doppeltbrechenden Kristall ist es nun so, dass die Phasengeschwindigkeiten für rechts- und linkszirkularpolarisiertes Licht unterschiedlich ist, was zur Polarisationsdrehung um den Winkel $\Theta$ führt. Für eine Welle welche an bei $z=0$ eintritt und in $x$-Richtung polarisiert ist
\begin{equation}
  E(0)=E_0\vec{x_0}\,,
\end{equation}
lässt sich mit
\begin{equation}
  \psi:=\frac{1}{2}\left(k_\text{R})+k_\text{L})
\end{equation}
und
\begin{equation}
\Theta:=\frac{1}{2}\left(k_\text{R})-k_\text{L})
\end{equation}
zeigen, dass sie sich nach einer Länge L durch
\begin{equation}
E(L)=E_0 \exp{\text{i}\psi}\left(\cos(\Theta)\vec{x_0}+\sin(\Theta)\vec{y_0})\,,
\end{equation}
beschreiben lässt.\\
Die Ursache des doppeltbrechenden Verhaltens einiger Kristalle liegt in induzierten Dipolen im Kristall, welche durch das Feld der Strahlung erzeugt werden. Diese Verursachen eine makroskopische Polarisierung $\vec{P}$ des Kristalls
\begin{equation}
\vec{P}=\epsilon_0\chi\vec{E}\,,
\end{equation}
wobei $\epsilon_0$ die Influenzkonstante und $\chi$ die dieelektrische Suszeptibilität ist.
