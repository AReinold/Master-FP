\section{Theorie}
\subsection{Die effektive Masse}%<-jk......darf ich nicht vergessen zu  ändern
Die physikalische Beschreibung von Elektronen in einem Kristall lässt sich am besten durch die Betrachtung der unteren Bandkante des Leitungsbandes annähern. Es lässt sich dann die Elektronenenergie $\epsilon(\vec{k})$, wobei $\vec{k}$ der Wellenzahlvektor ist, in einer Taylorreihe zu:
\begin{equation}
\epsilon(\vec{k})=\epsilon\left(0\right)+\frac{1}{2}\sum_{i=1}^3\left(\frac{\partial\epsilon^2}{\partial k_i^2}\right)_{k=0}k_i^2+...\,,
\end{equation}
entwickeln.
Vergleicht man dies mit einem harmonischen Oszillator mit
\begin{equation}
  \epsilon=\frac{\hbar k^2}{2m}\,,
\end{equation}
so stellt man fest, dass die Größe:
\begin{equation}
m_i^*:=\frac{\hbar^2}{\left(\frac{\partial\epsilon^2}{\partial k_i^2}\right)_{k=0}}\,,
\end{equation}
die Dimension einer Masse hat. Sie wird auch als effektive Masse des Kristallelektrons bezeichnet.
Für hinreichend hohe Symmentrien des Kristalls sind die einzelnen $m_i^*$ alle gleich groß und das Elektron lässt sich wie ein freies Teilchen mit Masse $m_i^*$ behandeln.
\subsection{Zirkulare Doppelbrechung}
Optische Doppelbrechung bezeichnet die Rotation der Polarisationsebene von linear polarisiertem Licht beim Durchqueren eines Mediums. Physikalisch lässt sich dieses Phänomen nachvollziehen, indem man linear polarisiertes Licht in zwei entgegengesetzt zirkular polarisierte Komponenten zerlegt
\begin{equation}
E(z)=\frac{1}{2}(E_\text{L}(z)+E_\text{R}(z))\,.
\end{equation}
In einem doppeltbrechenden Kristall ist es nun so, dass die Phasengeschwindigkeiten für rechts- und linkszirkular polarisiertes Licht unterschiedlich ist, was zur Polarisationsdrehung um den Winkel $\Theta$ führt. Für eine Welle, welche bei $z=0$ eintritt und in $x$-Richtung polarisiert ist
\begin{equation}
  E(0)=E_0\vec{x_0}\,,
\end{equation}
lässt sich mit
\begin{equation}
  \psi:=\frac{1}{2}\left(k_\text{R}+k_\text{L}\right)
\end{equation}
und
\begin{equation}
\Theta:=\frac{1}{2}\left(k_\text{R}-k_\text{L}\right)
\label{eq:theta}
\end{equation}
zeigen, dass sie sich nach einer Länge L durch
\begin{equation}
E(L)=E_0 \exp{\text{i}\psi}\left(\cos(\Theta)\vec{x_0}+\sin(\Theta)\vec{y_0}\right)\,,
\end{equation}
beschreiben lässt.\\
Die Ursache des doppeltbrechenden Verhaltens einiger Kristalle liegt in induzierten Dipolen im Kristall, welche durch das Feld der Strahlung erzeugt werden. Diese verursachen eine makroskopische Polarisierung $\vec{P}$ des Kristalls
\begin{equation}
\vec{P}=\epsilon_0\chi\vec{E}\,,
\end{equation}
wobei $\epsilon_0$ die Influenzkonstante und $\chi$ die dieelektrische Suszeptibilität ist.
Für anisotrope Kristalle ist $\chi$ ein Tensor der Form
\begin{equation}
  \chi=
  \left[ {\begin{array}{ccc}
   \chi_\text{xx} & \chi_\text{xy} & \chi_\text{xz}\\
   \chi_\text{yx} & \chi_\text{yy} & \chi_\text{yz}\\
   \chi_\text{zx} & \chi_\text{zy} &\chi_\text{zz}
  \end{array} } \right]\,.
\end{equation}
Während sich hier $\chi$ in vielen Fällen diagonalisieren lässt, treten für doppeltbrechende Materialien nicht-diagonale Elemente auf, welche komplex konjugiert zueinander sind.
\begin{equation}
  \chi=
  \left[ {\begin{array}{ccc}
   \chi_\text{xx} & \text{i}\chi_\text{xy} & 0\\
   \text{i}\chi_\text{yx} & \chi_\text{yy} & 0\\
   0 & 0 &\chi_\text{zz}
  \end{array} } \right]\,.
\end{equation}
Durch Lösen der Wellengleichung für eine ebene Welle in $\vec{z}$-Richtung lässt sich zeigen, dass die Wellenzahl für doppeltbrechende Medien nur die Werte
\begin{equation}
k_{\pm}=\frac{\omega}{\text{c}}\sqrt{(1+\chi_\text{xx})\pm \chi_{\text{xy}}}
\end{equation}
annehmen kann, wobei $\omega$ die Kreisfrequenz ist. Daraus folgt, dass zwei Phasengeschwindigkeiten möglich sind: Eine für rechtszirkular polarisiertes und eine für linkszirkular polarisiertes Licht
\begin{equation}
  \centering
v_{{\text{Ph}_\text{R}}}=\frac{\text{c}}{\sqrt{1+\chi_\text{xx}+\chi_\text{xy}}} \quad \text{und}\quad v_{{\text{Ph}_\text{L}}}=\frac{\text{c}}{\sqrt{1+\chi_\text{xx}-\chi_\text{xy}}} \,.
\end{equation}
Nach \eqref{eq:theta} ist es dann möglich den Drehwinkel mit
\begin{equation}
  \theta \approx \frac{\text{L}\omega}{2\text{c}n}\chi_\text{xy}
\end{equation}
zu bestimmen, wobei $n$ der Brechungsindex ist.
\subsection{Der Faraday-Effekt}
Der Faraday-Effekt beschreibt das Erzeugen von Doppelbrechung in einem an sich optisch inaktiven Medium durch das Anlegen eines äußeren Magnetfelds. Das Magnetfeld beeinflusst die Kristallelektronen, sodass die Bewegungsgleichung durch
\begin{equation}
  m\frac{\text{d}^2\,\vec{r}}{\text{d}\,t^2}+K\vec{r}=-\text{e}_0\left(\vec{E}(r)+\frac{\text{d}\,\vec{r}}{\text{d}\,t}\times\vec{B}\right)
\end{equation}
gegeben ist. Hierbei ist $\vec{r}$ die Auslenkung eines Elektrons aus dem Gleichgewicht, $K$ eine Bindungskonstante zur Umgebung, $m$ die Masse, e$_0$ die Ladung und $\vec{E}$ die elektrische Feldstärke.
Unter der Annahme, dass für die Feldstärke $E\sim\exp(-i\omega t)$ gilt und mit der Polarisation $\vec{P}=-N\text{e}_0\vec{r}$, wobei $N$ die Zahl der Elektronen pro Volumeneinheit ist, wird aus der Bewegungsgleichung
\begin{equation}
-m\omega^2\vec{P}+K\vec{P}=\text{e}_0^2N\vec{E}+i\text{e}_0\omega\vec{P}\times\vec{B}\,.
\end{equation}
Bei Betrachtung eines in z-Richtung liegenden Magnetfelds stellt sich heraus, dass der Suszeptibilitätstensor nicht-diagonale Elemente enthält, die komplex konjugiert zueinander sind.
Es lässt sich zeigen, dass der Drehwinkel der auftretenden Doppelbrechung durch
\begin{equation}
\theta=\frac{\text{e}_0^3\omega^2NBL}{2\epsilon_0cm^2\left(\left(-\omega^2+K/m\right)^2-\left(\frac{\text{e}_0}{m}B\omega\right)^2\right)n}
\end{equation}
beschrieben wird. Der Faktor $\sqrt{K/m}$ lässt sich als Resonanzfrequenz $\omega_0$ definieren und $\frac{B\text{e}_0}{m}$ als Zyklotron-Frequenz $\omega_c$. Für ein System, bei welchem die Messfrequenz viel kleiner als $\omega_0$ ist und $\omega_0>\omega_c$ ist, gilt:
\begin{equation}
\theta(\lambda)=\frac{2\pi^2\text{e}_0^3c}{\epsilon_0}\frac{1}{m^2}\frac{1}{\lambda^2\omega_0^4}\frac{NBL}{n}
\label{eq:theta2}
\end{equation}
Ersetzt man die Elektronenmasse $m$ durch die effektive Masse $m^*$, so lassen die Kristallelektronen sich wie freie Teilchen behandeln und (\ref{eq:theta2}) vereinfacht sich zu
\begin{equation}
\theta_\text{frei}=\frac{\text{e}_0^3}{8\pi^2\epsilon_0c^3}\frac{1}{{m^*}^2}\lambda^2\frac{NBL}{n}\,.
\end{equation}
