\section{Diskussion}
In diesem Kapitel sollen nun die erhaltenen Ergebnisse der Auswertung diskutiert werden.
Im Generellen kann zu der Durchführung des Versuchs gesagt werden, dass der Versuchstag frühzeitig abgebrochen werden musste, da die Versorgungsspannung zunehmender Schwankung unterlag. Hierdurch stieg der Wert der Betriebsspannung bei Weitem über den des Maximum Ratings, das zu einem Zerstören der Operationsverstärkerbauteile geführt hat.
Bei einer weiteren Diskussion soll beachtet werden, dass nicht ausgeschlossen werden kann, dass es bei der Aufnahme der Messdaten ebenfalls zu kleinen Schwankungen der Versorgungsspannung gekommen ist, sodass sich das Verstärkerverhalten unkontrolliert änderte. Des Weiteren ist hierüber und aufgrund des gebrauchten Zustands des Breadboards nicht sicher zu stellen, dass der Abschluss gegen Masse ausreichend ordentlich funktionierte, wodurch es ebenso zu Schwankungen im Ausgangssignal kommen konnte.\\
Betrachtet man die Messergebnisse des Versuchsteils mit den Linearverstärkeraufbauten, so können mehrere Punkte festgestellt werden. Darüber, dass das Widerstandsverhältnis nicht einheitlich $|>1|$ gewählt worden ist, können leichte Fehler durch das Umrechnen mit der Annahme eines idealen Operationsverstärkerverhalten aufgetreten sein.
Explizit zeigt sich in Abbildung \ref{linear1} ein stark unterschiedlicher Verlauf der Verstärkung im Gegensatz zu den anderen drei Verläufen \ref{linear2}, \ref{linear3} und \ref{linear4} im Bereich der Grenzfrequenz. Der Grund für das starke Ausreißen um diesen Frequenzpunkt kann nur gemutmaßt und auf Schwankungen in der Betriebsspannung oder ein kritisches Widerstandsverhältnis geschoben werden. Bis auf kleinere ausreißende Werte entspricht der erhaltene Verlauf für die anderen drei Verstärkerschaltungen aber den Erwartungen. Schaltung \ref{linear4} ist die einzige Schaltung, die mit einer Verstärkung $|>1|$ vermessen worden ist. Hier zeigt sich auch der Verlauf, der am ehesten der Theoriekurve entspricht. Ebenso sind hier die wenigsten Messwerte entstanden, die für die Ausgleichsrechnung ignoriert werden mussten.
Für ebendiese Schaltung ist auch das Phasenverhaltung zwischen Eingangs- und Ausgangsspannung beobachtet worden. Hier folgt die Messwertverteilung der Erwartung, bei der die Differenz von $\varphi=180°$ ab einer Frequenz von ca. $\SI{10}{kHz}$ abgebaut wird. Für ein weiteres Durchführen dieses Versuchteils wäre es von Vorteil, eine größere Widerstandsauswahl zur Verfügung zu haben, um einen größeren und diverseren Verstärkungsbereich untersuchen zu können.\\
Wird der Teil der Integrator- und Differentiatorschaltungen betrachtet, so ergibt sich insgesamt eine gute Übereinstimmung mit den erwarteten Werten. So zeigen die verschiedenen Thermodrucke beispielsweise deutlich die integrierenden und differenzierenden Eigenschaften der Schaltungen.\\
Auch beim Schmitt-Trigger ist die Abweichung zum Theoriewert mit 7$\%$ gering. Explizit kann hier festgestellt werden, dass um ein Schaltverhalten feststellen zu können, eine Eingangsspannung von $U_1=\SI{8,3}{V}$ angelegt werden musste. Da dies ein sehr hoher Wert ist, ist zu vermuten, dass aufgrund des folgenden Aussteigens des Aufbaues es hier schon zu unkontrollierten Aussetzern gekommen ist.
