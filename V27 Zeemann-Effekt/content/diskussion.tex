\section{Diskussion}
Festgestellt werden kann im Generellen, dass die experimentell ausgerechneten Landéschen g-Faktoren gut mit den Theoriewerten übereinstimmen. Für die rote Linie konnte leider aufgrund eines experimentellen Fauxpas keine weitere Auswertung stattfinden.
Für die $\sigma$-Linie ergab sich eine Abweichung von $\SI{14}{\%}$; für die $\pi$-Linie eine von $\SI{1,7}{\%}$. Zurückzuführen ist dies zu kleinen Teilen auf die Auswertung mittels des Programms \textit{Fiji}, das unter Umständen die Intensitätsmaxima nicht immer korrekt ausgelesen hat. Im Vergleich zu den aufgenommenen Bildern kann dies jedoch nicht für die Größe der Abweichungen verantwortlich sein. Vielmehr ist mit einem konstanten Zoom von $30x$ gearbeitet worden, sodass dieser für eine weitere Auswertung nicht herausgerechnet werden musste. Bei der Justierung des Strahlengangs und für die Aufnahme gut erkennbarer Bilder ist die Kamera mehrmals leicht nachjustiert worden, da diese auf ihrer Halterung aufgrund ihres Gewichts stets nach unten kippte. Wegen des hohen Vergrößerungsfaktors ist anzunehmen, dass hierdurch keine konstante Skalierung aufgenommen werden konnte.
