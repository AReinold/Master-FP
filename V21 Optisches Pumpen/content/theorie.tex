\section{Theorie}
\subsection{Der Zeeman-Effekt}
Verursacht durch den Elektronenspin $\vec{S}$ und den Bahndrehimpuls $\vec{L}$, hat jedes Energieniveau der Elektronen im Atom einen Gesamtdrehimpuls $\vec{J}$, welcher an ein magnetisches Moment koppelt:
\begin{equation}
  \vec{\mu}_J=\vec{\mu}_L+\vec{\mu}_S=-g_J\upmu_B\vec{J}
\end{equation}
bzw.
\begin{equation}
  ||\vec{\mu}_J||=-g_J\text{\mu}_B\sqrt{J(J+1)}\,.
\end{equation}
Hierbei ist $\upmu_B$ das bohrsche Magneton und $g_J$ der Landé-Faktor
\begin{equation}
  g_J=1+\frac{J+(J+1)+S(S+1)-L(L+1)}{2J(J+1)}
\end{equation}
Die magnetischen Momente $\mu_L$ und $\mu_S$ sind dabei gegeben durch
\begin{equation}
\vec{\mu}_L=-\mu_B\vec{L} \quad\quad\text{und}\quad\quad \vec{\mu}_S=-\text{g}_S\mu_B\vec{S}
\end{equation}
mit dem Landé-Faktor des freien Elektrons $\text{g}_S=2{,}00232$.
Wird das Atom zusätzlich von einem externen Magnetfeld$\vec{B}$ umgeben, so Wechselwirkt dieses mit dem Elektron und es tritt eine Aufspaltung der Energieniveaus auf. Die Wechselwirkungsenergie ist dabei gegeben durch
\begin{equation}
  W=g_J\upmu_BBM\,,
\end{equation}
wobei berücksichtigt wurde, dass sich die zu $\vec{B}$ senkrechte Komponente zeitlich herausmittelt und die  Richtungsquantelung nur ganzzahlige werte von $M \in [-J,J]$ zulässt. Es entsteht somit eine Aufspaltung jedes Energieniveaus in $2J+1$ Unterniveaus.
\subsection{Die Hyperfeinstrukturaufspaltung}
Neben dem 
