\section{Auswertung}
Zuerst wird die Stärke des gesamten Horizontal-Magnetfeldes aus der Abhängigkeit mit der Resonanzfrequenz für die beiden Rubidium-Isotope \ce{^{85}Rb} und \ce{^{87}Rb} bestimmt.
Die Magnetfeldstärke $B$ ist aus der abgelesenen Stromstärke $I$ mithilfe des Zusammenhangs
\begin{equation}
B=\mu_0\frac{8IN}{\sqrt{125}R}
\end{equation}
berechnet worden.
Für einen Vergleich mit der vertikalen Komponente des Erdmagnetfeldes ist der Versuchsaufbau parallel zur, mit einem Kompass bestimmten, Richtung des Erdmagnetfeldes ausgerichtet worden.
Über eine Vertikalfeldspule ist zusätzlich ein Magnetfeld der Größe
\begin{equation*}
B_\text{V}=0,229???
\end{equation*}
angelegt worden, um das Erdmagnetfeld zu kompensieren.
Bemerkbar ist es durch das schmaler Werden des Nullpunktpeaks geworden.
Abbildung \ref{fig:typSignalbild} zeigt einen typischen Signalverlauf. Zu sehen ist der tiefere Nullpunktspeak und die beiden Resonanzpeaks der Rubidium-Isotope.
\begin{figure}[H]
  \centering
  \includegraphics[width=0.5\textwidth]{Bilder/typSignalbild.jpg}
  \caption{Fotografie eines typischen Signalverlaufs bei \SI{100}{kHz}.}
  \label{fig:typSignalbild}
\end{figure}
In Tabelle \ref{tab:messwerte_resonanz} sind die gemessenen Positionen der Resonanzen in Abhängigkeit von der Frequenz $f_\text{RF}$ dargestellt.
\begin{table}[h]
  \centering
  \caption{Resonanzpositionen abhängig von der RF-Frequenz.}
  \label{tab:messwerte_resonanz}
  \begin{tabular}{S[table-format=1.1] S S S S}
    {$f_\text{RF}$ / MHz} & {$I_1$ / A} & {$I_2$ / A} & {$B_1$ / µT} & {$B_2$ / µT} \\
    \midrule
    0.1 &  0,035 &   0,044 & 4,85 & 6,09\\
    0.2 &  0,052 &   0,069 & 7,20 & 9,55\\
    0.3 &  0,069 &   0,095 & 9,55 & 13,15\\
    0.4 &  0,086 &  0,120 & 11,20 & 16,61\\
    0.5 &  0,204 &  0,330 & 28,24 & 45,70\\
    0.6 & 0,234 &  0,372 & 32,40 & 51,51\\
    0.7 & 0,273 &  0,447 & 37,80 & 61,70\\
    0.8 & 0,324 &  0,522 & 44,90 & 72,28\\
    0.9 & 0,372 &  0,600 & 51,51 & 83,08\\
    1.0 & 0,288 &  0,537 & 40,00 & 74,36\\
  \end{tabular}
\end{table}
