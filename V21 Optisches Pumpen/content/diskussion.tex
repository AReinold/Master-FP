\section{Diskussion}
Der erste Teil dieses Versuchs zum Verfahren des optischen Pumpens handelte von der Kompensation des Erdmagnetfeldes.
Bei Betrachtung fällt auf, dass die Werte des Sweep-Feldes für beide Resonanzstellen nicht beachtlich voneinander abweichen. Dorthinzukommend dominiert das benötigte horizontale Magnetfeld, das den Sweep-Bereich auf die Resonanzen verschieben soll.
Es zeigt sich, dass der Versuchsteil dorthingehend unbrauchbar geworden ist, da die erhaltenen Werte so eng beieinander liegen, dass Rückschlüsse auf die Isotopenzuordnung nicht gezogen werden können. Ebenso die aus dieser Messung folgenden Berechnungen der Landé-Faktoren und des Kernspins sowie die Berechnung des Erdmagnetfeldes sind wenig aufschlussreich, trotzdessen, dass die Peaks auf dem Oszilloskop klar erkennbar waren.\\
Für eine Berechnung des Isotopenverhältnisses liegen die herausgefundenen Werte um $26,3\%$ von der in der Natur vorkommenden Konzentration entfernt.
Aufgrunddessen, dass es sich um ein Laborgemisch handelt, spiegelt die Messung einen erwartbaren Wert wider.
Dass das Ablesen aufgrund der ungenauen Drehregler und das anschließende Umrechnen Grund dafür ist, kann aufgrund der hohen benötigten Kompensationsmagnetfeldstärke ausgeschlossen werden. Demnach ist darauf zu schließen, dass äußere lokale Magnetfelder im Versuchsraum die Messung gestört haben. Während des Versuchs ist aufgefallen, dass die Regler für die Kompensation stetig nachgeregelt werden mussten und dauernden Schwankungen unterlagen. Dies unterstützt die Hypothese.\\
Für die Untersuchung der transienten Effekte ist auffällig, dass sich die erwartbaren Funktionsverläufe eingestellt haben.
Die Messwerte zeigen das typische Verhalten für beide Resonanzstellen, die allerdings zuvor keinem direkten Isotop zugeordnet werden konnten. Eine Abweichung des Verhältnisses zwischen den beiden Landé-Faktoren und dem Literaturwert von $69,33\%$ ist zunächst als groß anzusehen, liegt aber im Vergleich zu den vorherigen Messungen in einem guten Rahmen.
