\section{Diskussion}
Aus dem ersten Auswertungteil ließen sich die Landéschen Faktoren experimentell bestimmen und Rückschlüsse auf die Zuordnung zu dem passenden Isotop schließen. Dieses Zuordnen lieferte eine Abweichung im Kernspin $I$ von $0,8\%$ für $\ce{^{87}Rb}$ und $0,32\%$ für $\ce{^{85}Rb}$. Somit können die Resonanzpeaks eindeutig zugeordnet werden.
Für eine Berechnung des Isotopenverhältnisses liegen die herausgefundenen Werte um $26,3\%$ von der in der Natur vorkommenden Konzentration entfernt.
Aufgrund dessen, dass es sich um ein Laborgemisch handelt, spiegelt die Messung einen erwartbaren Wert wider.
Das Verhältnis der Landéschen Faktoren beträgt in der Theorie $1,5$. Das hier experimentell bestimmte Verhältnis beträgt $1,514$. Dies entspricht einer Abweichung von $0,92\%$.
Für die Untersuchung der transienten Effekte ist auffällig, dass sich die erwartbaren Funktionsverläufe eingestellt haben.
Die Messwerte zeigen das typische Verhalten für beide Resonanzstellen, die allerdings zuvor keinem direkten Isotop zugeordnet werden konnten. Eine Abweichung des Verhältnisses zwischen den beiden Landé-Faktoren und dem Literaturwert von $69,33\%$ ist zunächst als groß anzusehen, liegt aber im Vergleich zu den vorherigen Messungen in einem guten Rahmen.
